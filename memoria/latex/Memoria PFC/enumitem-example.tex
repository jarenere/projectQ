%% LyX 2.1.0 created this file.  For more info, see http://www.lyx.org/.
%% Do not edit unless you really know what you are doing.
\documentclass[a4paper,british]{article}
\usepackage{lmodern}
\renewcommand{\sfdefault}{lmss}
\renewcommand{\ttdefault}{lmtt}
\usepackage[T1]{fontenc}
\usepackage[utf8]{luainputenc}
\usepackage{babel}
\usepackage[unicode=true]
 {hyperref}

\makeatletter

%%%%%%%%%%%%%%%%%%%%%%%%%%%%%% LyX specific LaTeX commands.
\pdfpageheight\paperheight
\pdfpagewidth\paperwidth

\providecommand{\LyX}{\texorpdfstring%
  {L\kern-.1667em\lower.25em\hbox{Y}\kern-.125emX\@}
  {LyX}}

%%%%%%%%%%%%%%%%%%%%%%%%%%%%%% Textclass specific LaTeX commands.
\usepackage{enumitem}		% customizable list environments
\newlength{\lyxlabelwidth}      % auxiliary length 
% labeling-like list based on enumitem's description list with
% mandatory second argument (label-pattern):
\newenvironment{elabeling}[2][]%
{\settowidth{\lyxlabelwidth}{#2}
\begin{description}[font=\normalfont,style=sameline,
leftmargin=\lyxlabelwidth,#1]}
{\end{description}}

\makeatother

\begin{document}

\section*{The enumitem Module}

The enumitem Module provides customisable list Styles using the \href{http://dante.ctan.org/CTAN/help/Catalogue/entries/enumitem.html}{enumitem LaTeX package}.


\subsection*{Lists with optional arguments}

Itemize, Enumeration, and Description lists may have an optional arguments.
If the optional argument contains special characters (e.g. the backslash),
put it in an ERT box.
\begin{enumerate}[labelindent=\parindent,leftmargin=*,label=\Roman*.,widest=IV,align=left ]
\item  An enumeration
\item with left-aligned roman
\item numbering
\item 1of items.
\end{enumerate}
Enumerating with ``Spanish layout'': italic letters followed by
)
\begin{enumerate}[label=\emph{\alph*})]
\item  first item
\item second item
\end{enumerate}
Enumeration starting at a given value:
\begin{enumerate}[start=4]
\item  This enumeration
\item starts at 4.
\end{enumerate}
Description with emphasized (instead of bold) label and  left margin
1.5 em:
\begin{description}[leftmargin=1.5em,font=\itshape\mdseries]
\item [{Strahlungsmodulation:}] Durch die zeitliche Modulation der auf
den Detektor treffenden Strahlung wird trotz fehlender Gleichlichtempfindlichkeit
der pyroelektrischen Detektoren die reproduzierbare Betrachtung statischer
Wärmeszenen ermöglicht.
\item [{Referenzstrahlung:}] Die Wärmestrahlung des Choppers geht direkt
in das Messsignal ein.
\end{description}
More details and examples in \href{http://dante.ctan.org/CTAN/macros/latex/contrib/enumitem/enumitem.pdf}{enumitem.pdf}.


\subsection*{Resume enumeration}



Enumerations can be resumed after intermediate paragraphs:
\begin{enumerate}
\item first
\item second
\end{enumerate}
regular text
\begin{enumerate}[resume]
\item resumed (the numbering is not continued in the \LyX{} GUI but correct
in the \LaTeX{} output).
\end{enumerate}
Nesting enumerations:
\begin{enumerate}
\item depth 1

\begin{enumerate}
\item first

\begin{enumerate}[resume]
\item resume without something to resume! (Works here.)
\end{enumerate}
\item second

\begin{enumerate}
\item with something nested inside
\end{enumerate}
\end{enumerate}

regular text 
\begin{enumerate}[resume]
\item resumed at depth 2
\end{enumerate}
\end{enumerate}
regular text outside the list


\subsection*{Compact Lists}

There are keywords for more compact variants of the standard lists:
\begin{itemize}
\item A bullet list
\item with standard spacing 
\end{itemize}
lorem ipsum
\begin{itemize}[nolistsep]
\item A bullet list 
\item with less 
\item vertical space. (nolistsep)
\end{itemize}
lorem ipsum
\begin{itemize}[noitemsep]
\item A bullet list 
\item with less 
\item vertical space. (noitemsep)
\end{itemize}
You can customize the itemset and listsep for all base lists in the
\LaTeX{} preamble. (Also per list type.) See \href{http://dante.ctan.org/CTAN/macros/latex/contrib/enumitem/enumitem.pdf}{enumitem.pdf}.


\subsection*{Customisable \protect\LyX{}-List/Labeling}

The \LyX{} ``List'' style (or KOMA script ``labeling'') is re-defined
on the base of enumitems ``description'' to allow customisation
like the base lists.
\begin{elabeling}{00.00.0000}
\item [{nice}] description 
\item [{with}] several items and
\item [{including~one~very~long~label}] at last. 
\end{elabeling}
It also takes an optional argument:
\begin{elabeling}{withNN}
\item [{nice}] description 
\item [{with}] several items and
\item [{including~one~very~long~label}] and the item content starting
at the next line.\end{elabeling}

\end{document}
