%% LyX 2.1.0 created this file.  For more info, see http://www.lyx.org/.
%% Do not edit unless you really know what you are doing.
\documentclass[12pt,spanish,a4paper,titlepage]{ociamthesis-lyx}
\usepackage[T1]{fontenc}
\usepackage[utf8]{luainputenc}
\usepackage{fancyhdr}
\pagestyle{fancy}
\setcounter{secnumdepth}{3}
\setcounter{tocdepth}{3}
\usepackage{setspace}
\usepackage[numbers]{natbib}
\onehalfspacing

\makeatletter
%%%%%%%%%%%%%%%%%%%%%%%%%%%%%% Textclass specific LaTeX commands.
\newenvironment{lyxlist}[1]
{\begin{list}{}
{\settowidth{\labelwidth}{#1}
 \setlength{\leftmargin}{\labelwidth}
 \addtolength{\leftmargin}{\labelsep}
 \renewcommand{\makelabel}[1]{##1\hfil}}}
{\end{list}}

\makeatother

\usepackage{babel}
\addto\shorthandsspanish{\spanishdeactivate{~<>}}

\begin{document}

\chapter{Esqueleto de la aplicación construida}

En este apartado vamos a indicar el esqueleto sobre el que hemos construido
la aplicación, haciendo uso de las distintas tecnologías mencionadas
en el contexto tecnológico. 


\section{Flask}

Siguiendo los distintos consejos dados en la documentación de Flask,
así como las recomendaciones de patrones para Flask, la estructura
de nuestra plataforma de encuestas es la siguiente: 

\setlength{\itemsep}{0mm} 
\begin{description}
\item[SwarmSurvey] -Directorio raíz de la aplicación
\item[Lo segundo] adarna antigua,
\begin{description}
\item[Lo tercero] rdf, y
\end{description}
\item[Y por fso, lo cuar] galgo corredor.
\end{description}
\begin{lyxlist}{00.00.0000}
\item [{}]~
\item [{SwarmSurvey}] -Directorio raíz de la aplicación
\item [{manage.py}] -Se encarga de la gestión de la aplicación, ya sea
ejecutar el servidor, migrar la base de datos, o entrar en la interfaz
de linea de comandos de la aplicación

\begin{lyxlist}{00.00.0000}
\item [{config.py}] -Fichero de configuración base, usando un modelo de
herencia de clases para la configuración, así como el uso de variables
de entorno para cambiar de configuración
\item [{app}]~

\begin{lyxlist}{00.00.0000}
\item [{auth}] -Módulo para el registro y autentificación de los usuarios

\begin{lyxlist}{00.00.0000}
\item [{\_\_init\_\_.py}] -Fichero de inicialización del módulo
\item [{forms.py}] -Formularios que se mostrarán en las plantillas para
el registro y autentificación de usuarios
\end{lyxlist}
\end{lyxlist}
\end{lyxlist}
\item [{validators.py}] -Fichero con los distintos validadores creados
para comprobar los formularios
\item [{views.py}] -Fichero con la resolución de rutas y las funciones
expuestas a través de estas rutas (eg: /login, /register...)
\item [{feedback}] -Módulo que da soporte de feedback al experimento \textquotedbl{}¿Cómo
son nuestros voluntarios?\textquotedbl{}
\item [{...}]~
\item [{funtion\_jinja}] -Módulo que contiene distintas funciones creadas
para el generador de plantillas jinja2
\end{lyxlist}
\_\_init\_\_.py

functions.py

game -Módulo que contiene la lógica a la hora de seleccionar los usuarios
para los juegos del experimento \textquotedbl{}¿Cómo son nuestros
voluntarios?\textquotedbl{}

\_\_init\_\_.py

game.py -Juegos del apartado 3 del experimento \textquotedbl{}¿Cómo
son nuestros voluntarios?\textquotedbl{}

game\_impatience.py -Juegos del apartado 2 del experimento \textquotedbl{}¿Cómo
son nuestros voluntarios?\textquotedbl{}

raffle.py -Rifa del experimento \textquotedbl{}¿Cómo son nuestros
voluntarios?\textquotedbl{}

main -Módulo que contiene la información básica de la plataforma

\_\_init\_\_.py 

errors.py -Personalización de los distintos errores que puede lanzar
la plataforma, ya sean por parte del cliente(4xx) o del servidor(5xx)

views.py -Vistas del la ruta raiz (/), así como el selector de idioma.

researcher -Módulo para la creación de encuestas por parte de los
investigadores

...

scheduler -Módulo para la planificación de tareas en el tiempo

...

surveys -Módulo para visualizar y contestar encuestas por parte de
los usuarios.

...

static -Directorio que contiene los elementos estáticos de la plataforma
web, como puede ser el uso de imágenes, css...

css

img

js

text

stats -Módulo que contiene la generación y visualización de estadísticas
y resultados de las encuestas así como del experimento \textquotedbl{}¿Cómo
son nuestros voluntarios?\textquotedbl{}

...

templates -Directorio con las distintas plantillas usadas para la
generación de las distintas páginas web

auth -Directorio con las plantillas del módulo auth

login.html -Plantilla de inicio de sesión a través de OpenID

register.html -Plantilla de registro

loginEmail -Plantilla de inicio a través de correo/contraseña

feedback

...

...

translations -Directorio con los distintos idiomas que soporta la
plataforma

es -Traducción al español

en -Traducción al ingles

\_\_init\_\_.py -Fichero de inicialización de la plataforma web

decorators.py -Fichero con las funciones para la protección de las
distintas vistas de la plataforma

models.py -Declaración del modelo de clases que hace uso de la base
de datos.

...

migrations -Directorio que contiene la información necesaria para
la migración de la base de datos, este directorio se genera automáticamente.

...

tests -Directorio que contiene los distintos tests unitarios del sistema

test\_models.py -Fichero con los test del modelo

...

logs -Directorio con los logs de la aplicación
\end{document}
